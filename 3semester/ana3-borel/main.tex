\documentclass{scrartcl}
\usepackage{microtype}
\usepackage{geometry}
\usepackage[T1]{fontenc}
\usepackage[utf8]{inputenc}
\usepackage[ngerman]{babel}
% Standard-Packages

\usepackage[utf8]{inputenc}
\usepackage[T1]{fontenc}
\usepackage{amssymb}
\usepackage[fleqn]{amsmath}
\usepackage{amsfonts}
\usepackage{enumerate}
\usepackage{microtype}
\usepackage{extpfeil}
\usepackage{ngerman}
\usepackage{gauss}
\usepackage{mathtools}
\usepackage{mathrsfs}
\usepackage{xcolor}[dvipsnames]
\definecolor{MyBlue}{rgb}{0.2,0.2,0.7}


\usepackage{listofitems}
\newcommand\cycle[2][\,]{%
  \readlist\thecycle{#2}%
  (\foreachitem\i\in\thecycle{\ifnum\icnt=1\else#1\fi\i})%
}

% amsthm-Formatierung und Änderungen
\usepackage{amsthm}
\theoremstyle{plain}
\newtheorem{kor}{Korrollar}
\newtheorem{satz}{Satz}
\newtheorem{lemma}{Lemma}
\renewcommand*{\proofname}{Beweis}
\theoremstyle{remark}
\newtheorem*{bem}{\textbf{Bemerkung}}
\theoremstyle{definition}
\newtheorem{defn}{Definition}
\newtheorem*{defnstar}{Definition}

\newtheoremstyle{erinnerung}
{3pt}% Space above
{3pt}% Space below 
{}% Body font
{}% Indent amount
{\bfseries}% Theorem head font
{.}% Punctuation after theorem head
{.5em}% Space after theorem head
{}% Theorem head spec (can be left empty, meaning ‘normal’)

\theoremstyle{erinnerung}
\newtheorem*{remind}{Erinnerung}

% Lineare Algebra 
\newcommand{\spur}{\operatorname{Sp}}
\newcommand{\ann}{\operatorname{Ann}}
\newcommand{\charPol}[1]{\chi_{#1}^\text{char}}
\newcommand{\minPol}[1]{\chi_{#1}^\text{min}}
\newcommand{\Hom}[1]{\mathrm{Hom}_{#1}}
\newcommand{\End}[1]{\mathrm{End}_{#1}}
\newcommand{\Spv}[1]{\operatorname{Spv}(#1)}
\newcommand{\Spec}[1]{\operatorname{Spec}(#1)}
\newcommand{\RZ}[1]{\operatorname{RZ}(#1)}
\newcommand{\Xmeas}{(X,\mathscr{E},\mu)}
\newcommand{\Ymeas}{(X,\mathscr{F},\nu)}
\newcommand{\legendre}[2]{\left(\frac{#1}{#2}\right)}
\DeclareMathOperator{\kgV}{kgV}
\DeclareMathOperator{\ggT}{ggT}
\DeclareMathOperator{\GGT}{GGT}
\DeclareMathOperator{\im}{im}
\DeclareMathOperator{\rg}{Rang}
\DeclareMathOperator{\GL}{GL}
\DeclareMathOperator{\sgn}{sgn}
\DeclareMathOperator{\Lin}{Lin}
\DeclareMathOperator{\id}{id}
\DeclareMathOperator{\esssup}{\operatorname{ess \ sup}}
\DeclareMathOperator{\pr}{pr}
\DeclareMathOperator{\gal}{Gal}
\DeclareMathOperator{\ord}{ord}
\DeclareMathOperator{\supp}{supp}

% Zahlenmengen
\newcommand{\R}{\mathbb{R}}
\newcommand{\Q}{\mathbb{Q}}
\newcommand{\Z}{\mathbb{Z}}
\newcommand{\C}{\mathbb{C}}
\newcommand{\K}{\mathbb{K}}
\newcommand{\N}{\mathbb{N}}
\renewcommand{\P}{\mathbb{P}}
\newcommand{\F}{\mathbb{F}}

% Formatierung & Sonstiges 
\newcommand{\D}{\mathrm{d}}
\renewcommand{\phi}{\varphi}
\renewcommand{\epsilon}{\varepsilon}

\newcommand{\cali}[1]{\mathcal{#1}}
\usepackage{tikz-cd}
% header and footer with scrlayer-scrpage
\usepackage{scrlayer-scrpage}
\usepackage{fleqn}
\usepackage{xcolor}

\lohead*{Höhere Analysis}
\rohead*{Nils Witt}
\cofoot*{}
\lofoot*{Seite \thepage}
\pagestyle{scrheadings}
\title{Borel-Mengen auf dem $\R^n$.}
\date{Wintersemester 2020}
\author{}
\setlength\parindent{0pt}

\begin{document}
    \maketitle Seien hier stets $X,X_1$ und $X_2$ nichtleere Mengen. 
    \begin{lemma}
        Sei $Y$ eine nichtleere Menge und $f:X\to Y$ eine Abbildung. Sei $\cali{E}$ eine $\sigma$-Algebra auf $X$ und $\cali{F}$ eine $\sigma$-Algebra auf $Y$. Dann sind 
        \[
        f^{-1}(\cali{F})=\{f^{-1}(F):F\in \cali{F}\}, \quad  f_*(\cali{E})=\{B\subset Y:f^{-1}(B)\in \cali{E}\}
        \]
        $\sigma$-Algebren.
    \end{lemma}
    \begin{proof}
        \begin{enumerate}[(a)]
            \item Es ist $f^{-1}(\emptyset) = \emptyset$, also $\emptyset\in f^{-1}(\cali{F})$. Jedes $A\in f^{-1}(\cali{F})$ können wir als $A=f^{-1}(F)$ für ein $F\in \cali{F}$ schreiben. Dann ist direkt klar, dass
            \[
            A^c =(f^{-1}(F))^c = f^{-1}(F^c)    
            \] wegen $F^c\in \cali{F}$ auch $A^c\in f^{-1}(\cali{F})$ ist. Seien noch $A_n\in f^{-1}(\cali{F})$ für $n\in \N$, so existieren $F_n \in \cali{F}$ mit $f^{-1}(F_n) = A_n$ für alle $n\in \N$. Folglich ist
            \[
            \bigcup_{n\in \N} A_n = \bigcup_{n\in \N}f^{-1}(F_n) = f^{-1}(\bigcup_{n\in \N}F_n)    
            \] und daher wegen $\bigcup_{n\in \N}F_n \in \cali{F}$ auch $\bigcup_{n\in \N}A_n\in f^{-1}(\cali{F})$.
        \item Da $\emptyset\in \cali{E}$ und $f^{-1}(\emptyset) = \emptyset$, ist auch $\emptyset\in f_*(\cali{E})$. Sei $A\in f_*(\cali{E})$, dann gilt
        \[
        f^{-1}(A^c) = f^{-1}(A)^c    
        \] und wegen $f^{-1}(A)\in \cali{E}$, ist auch $f^{-1}(A)^c\in \cali{E}$, also ist $A^c\in f_*(\cali{E})$. Seien $A_n\in f_*(\cali{E})$ für $n\in \N$. Dann ist
        \[
        f^{-1}(\bigcup_{n\in\N}A_n) = \bigcup_{n\in\N} f^{-1}(A_n)    
        \] und da $f^{-1}(A_n) \in \cali{E}, \ \forall n\in \N$, ist auch $\bigcup_{n\in\N} f^{-1}(A_n)\in \cali{E}$.
        \end{enumerate}
    \end{proof}
    \begin{defn}[Borelsche $\sigma$-Algebra und topolgischer Raum]
        Sei $(X,\mathfrak{T})$ ein topolgischer Raum. Das heißt $\mathfrak{T}\subset \mathfrak{P}(X)$ mit den Eigenschaften
        \begin{enumerate}[(1)]
            \item $X,\emptyset \in \mathfrak{T}$
            \item Für $A,B \in \mathfrak{T}$ gilt $A\cap B\in \mathfrak{T}$
            \item Sei $I\neq \emptyset$ eine beliebige nichtleere Indexmenge und $A_i\in \mathfrak{T}$, dann ist $\bigcup_{i\in I} A_i \in \mathfrak{T}$.
        \end{enumerate}
        Wir nennen $\sigma(\mathfrak{T})=\cali{B}(X)$ die Borel-$\sigma$-Algebra auf $X$.
    \end{defn}
    Seien $(X_j,\cali{A}_j)$ für $j=1,2$ messbare Räume. Wir definieren die Produkt-$\sigma$-Algebra wie folgt
    \[
    \cali{A}_1\otimes \cali{A}_2 \coloneqq \sigma(\{A\times B : A\in \cali{A}_1, B\in \cali{A}_2 \})   
    \] Ferner bezeichnen wir mit 
    \[
    \cali{A}_1 \boxtimes \cali{A}_2 \coloneqq \{A\times B : A\in \cali{A}_1, B\in \cali{A}_2\}    
    \]
    die Menge der kartesischen Produkte. Das nächste Lemma war das eigentliche Ziel, das ich dir zeigen wollte. Es ist eine Verallgemeinerung davon, dass $\cali{B}(\R)\times \cali{B}(\R) = \sigma(\{A_1\times A_2 : A_1,A_2\subset \R \ \text{offen}\})$. 
    Dass sich also die Produkt-$\sigma$-Algebra von den kartesischen Produkten von Erzeugern erzeugen lässt. Also bei uns war es, dass die Produkt-$\sigma$-Algebra die kleinste $\sigma$-Algebra ist, die 
    alle kartesischen Produkte enthält und falls es Erzeuger gibt, liefert uns das nachfolgende Lemma, dass die Produkt-$\sigma$-Algebra aus den kartesischen Produkten der Erzeuger erzeugt wird.
    \begin{lemma}
        Seien $\cali{S}_j\in \mathfrak{P}(X_j)$ Systeme von Teilmengen mit $X_j \in \cali{S}_j$ für $j=1,2$. So gilt 
        \[
            \sigma(\cali{S}_1) \otimes \sigma(\cali{S}_2) = \sigma(\cali{S}_1\boxtimes \cali{S}_2)
        \]
        \begin{proof}
            Wir definieren $\cali{A}_j = \sigma(\cali{S}_j)$ für $j=1,2$. Es gilt $\sigma(\cali{S}_1\boxtimes \cali{S}_2)\subset \sigma(\cali{S}_1)\otimes \sigma(\cali{S}_2)$, denn sei $S_1\times S_2\in \cali{S}_1\boxtimes \cali{S}_2$, dann gilt insbesondere, dass $S_1\in \sigma(\cali{S}_1)=\cali{A}_1$ und $S_2\in \sigma(\cali{S}_2)=\cali{A}_2$, also ist $S_1\times S_2 \in \{A\times B:A\in \cali{A}_1, B \in \cali{A}_2\}$ und daher nach Definition $S_1\times S_2\in \sigma(\{A\times B:A\in \cali{A}_1, B \in \cali{A}_2\})=\cali{A}_1\otimes \cali{A}_2$. \\
            Da $\cali{A}_1\otimes \cali{A}_2$ eine $\sigma$-Algebra ist, die $\cali{S}_1\boxtimes \cali{S}_2$ enthält, folgt die Behauptung. Noch die andere Inklusion. Wir definieren die $\sigma$-Algebren
            \[
            \widetilde{\cali{A}_j} \coloneqq (\mathrm{pr}_j)_* (\sigma(\cali{S}_1 \boxtimes \cali{S}_2))    
            \]
            wobei $\pr_j$ die Projektion von $X_1\times X_2$ auf den Raum $X_j$ meint, für $j=1,2$. Nun gilt \textit{nach Voraussetzung}, dass $X_j \in\cali{S}_j$ für $j=1,2$. Wir zeigen, dass $\cali{A}_j\subset \widetilde{\cali{A}_j}$ für $j=1,2$. Sei nun $j=1$ und sei $S_1 \in \cali{S}_1$ beliebig, dann ist 
            \[
                \pr_1^{-1}(S_1) = S_1 \times X_2 \in \cali{S}_1\boxtimes \cali{S}_2 \subset \sigma(\cali{S}_1\boxtimes \cali{S}_2)
            \] weil $S_1\in \cali{S}_1$ und $X_2\in \cali{S}_2$. Nach Definition ist 
            \[
            \cali{A}_1 = (\pr_1)_*(\sigma(\cali{S}_1\boxtimes \cali{S}_2)) = \{A\in X_1\times X_2 : \pr_1^{-1}(A)\in \sigma(\cali{S_1}\boxtimes\cali{S}_2)\}    
            \]Daher ist also $S_1\in (\pr_1)_*(\sigma(\cali{S}_1\boxtimes\cali{S}_2))=\widetilde{\cali{A}_1}$. Also ist $\widetilde{\cali{A}_1}$ eine $\sigma$-Algebra auf $X_1$, die $\cali{S}_1$ enthält, also ist $\sigma(\cali{S}_1) = \cali{A}_1 \subset \widetilde{\cali{A}_1}$. Analog für $j=2$.
            Daher haben wir, dass $\cali{A}_j \subset \widetilde{\cali{A}_j}$ für $j=1,2$. \\ Sei nun $A_1\times A_2\in \sigma(\cali{S}_1)\boxtimes \sigma(\cali{S}_2) = \cali{A}_1\boxtimes \cali{A}_2$. Dann gilt 
            \begin{align*}
                A_1\times X_2 &= (\pr_1)^{-1}(A_1) \in \sigma(\cali{S}_1\boxtimes \cali{S}_2) \\
                X_1 \times A_2 &= (\pr_2)^{-1}(A_2) \in \sigma(\cali{S}_1\boxtimes \cali{S}_2)
            \end{align*}
            weil gilt, dass $\sigma(\cali{S}_1) = \cali{A}_1 \subset  \widetilde{\cali{A}_1}= (\pr_1)_*(\sigma(\cali{S}_1\boxtimes\cali{S}_2))$ das bedeutet nach Definition nichts anderes, als, dass für alle $M_1\in \sigma(\cali{S}_1)=\cali{A}_1$ gilt, dass $\pr_1^{-1}(M_1)\in \sigma(\cali{S_1}\boxtimes \cali{S}_2)$. Aus analogen Gründen sieht man dann die zweite Zeile ein. \\
            Nun bemerkt man noch, dass $A_1\times A_2 = (A_1\times X_2)\cap (X_1\times A_2)$. Ferner gilt, dass $A_1\times X_2,X_1\times A_2\in \sigma(\cali{S}_1\boxtimes\cali{S}_2)$, da die rechte Menge eine $\sigma$-Algebra ist, gilt auch, dass $A_1\times A_2\in \sigma(\cali{S}_1\boxtimes\cali{S}_2)$. Daher gilt, dass $\cali{A}_1\boxtimes \cali{A}_2 \subset \sigma(\cali{S}_1\boxtimes\cali{S}_2)$. Also gilt 
            \[
            \cali{A}_1\otimes \cali{A}_2 = \sigma(\cali{A}_1\boxtimes \cali{A}_2) \subset \sigma(\cali{S}_1\boxtimes \cali{S}_2)    
            \] was zu zeigen war.
        \end{proof}
    \end{lemma}
    Sei nun $(X,\cali{E},\mu)$ ein Maßraum.
    \begin{lemma}[Chebyshev'sche-Ungleichung]
        Sei $f:X\to [0,\infty]$ messbar, dann gilt für $t\in (0,\infty)$, dass
        \[
        t\cdot \mu(\{x\in X: f(x)>t\}) \le \int_X f \ \D \mu    
        \] 
    \end{lemma}
    Wir zeigen damit, dass für $f:X\to [0,\infty]$ messbar gilt
    \[
    \int_X f \ \D \mu < \infty \implies f<\infty \ \mu\text{-fast-überall}     
    \]
    Angenommen $f$ wäre nicht $\mu$-fast-überall endlich, dann gäbe es eine Menge $A\in \cali{E}$ mit $\mu(E)>0$ und $f(x)=\infty$ für alle $x\in E$. Dann wäre aber für alle $t>0$
    \[
    \mu(\{x\in X: f(x)\ge t\}) \ge \mu(E)    
    \]
    da für alle $x\in E$ gilt, $f(x)=\infty > t$. Also gilt für alle $n\in \N$, dass 
    \[
    n \cdot \mu(E) \le n\cdot\mu(\{x\in X: f(x)\ge n)\} \le \int_X f \ \D \mu    
    \]
    aber, da $\mu(E)$ konstant ist, gilt $n\cdot \mu(E) \to \infty$, wenn $n\to \infty$. Also kann $\int_X f\ \D \mu$ nicht endlich sein. Widerspruch.
\end{document}