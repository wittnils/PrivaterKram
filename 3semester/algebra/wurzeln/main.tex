\documentclass{scrartcl}
\usepackage{microtype}
\usepackage{geometry}
\usepackage[T1]{fontenc}
\usepackage[utf8]{inputenc}
\usepackage[ngerman]{babel}
% Standard-Packages

\usepackage[utf8]{inputenc}
\usepackage[T1]{fontenc}
\usepackage{amssymb}
\usepackage[fleqn]{amsmath}
\usepackage{amsfonts}
\usepackage{enumerate}
\usepackage{microtype}
\usepackage{extpfeil}
\usepackage{ngerman}
\usepackage{gauss}
\usepackage{mathtools}
\usepackage{mathrsfs}
\usepackage{xcolor}[dvipsnames]
\definecolor{MyBlue}{rgb}{0.2,0.2,0.7}


\usepackage{listofitems}
\newcommand\cycle[2][\,]{%
  \readlist\thecycle{#2}%
  (\foreachitem\i\in\thecycle{\ifnum\icnt=1\else#1\fi\i})%
}

% amsthm-Formatierung und Änderungen
\usepackage{amsthm}
\theoremstyle{plain}
\newtheorem{kor}{Korrollar}
\newtheorem{satz}{Satz}
\newtheorem{lemma}{Lemma}
\renewcommand*{\proofname}{Beweis}
\theoremstyle{remark}
\newtheorem*{bem}{\textbf{Bemerkung}}
\theoremstyle{definition}
\newtheorem{defn}{Definition}
\newtheorem*{defnstar}{Definition}

\newtheoremstyle{erinnerung}
{3pt}% Space above
{3pt}% Space below 
{}% Body font
{}% Indent amount
{\bfseries}% Theorem head font
{.}% Punctuation after theorem head
{.5em}% Space after theorem head
{}% Theorem head spec (can be left empty, meaning ‘normal’)

\theoremstyle{erinnerung}
\newtheorem*{remind}{Erinnerung}

% Lineare Algebra 
\newcommand{\spur}{\operatorname{Sp}}
\newcommand{\ann}{\operatorname{Ann}}
\newcommand{\charPol}[1]{\chi_{#1}^\text{char}}
\newcommand{\minPol}[1]{\chi_{#1}^\text{min}}
\newcommand{\Hom}[1]{\mathrm{Hom}_{#1}}
\newcommand{\End}[1]{\mathrm{End}_{#1}}
\newcommand{\Spv}[1]{\operatorname{Spv}(#1)}
\newcommand{\Spec}[1]{\operatorname{Spec}(#1)}
\newcommand{\RZ}[1]{\operatorname{RZ}(#1)}
\newcommand{\Xmeas}{(X,\mathscr{E},\mu)}
\newcommand{\Ymeas}{(X,\mathscr{F},\nu)}
\newcommand{\legendre}[2]{\left(\frac{#1}{#2}\right)}
\DeclareMathOperator{\kgV}{kgV}
\DeclareMathOperator{\ggT}{ggT}
\DeclareMathOperator{\GGT}{GGT}
\DeclareMathOperator{\im}{im}
\DeclareMathOperator{\rg}{Rang}
\DeclareMathOperator{\GL}{GL}
\DeclareMathOperator{\sgn}{sgn}
\DeclareMathOperator{\Lin}{Lin}
\DeclareMathOperator{\id}{id}
\DeclareMathOperator{\esssup}{\operatorname{ess \ sup}}
\DeclareMathOperator{\pr}{pr}
\DeclareMathOperator{\gal}{Gal}
\DeclareMathOperator{\ord}{ord}
\DeclareMathOperator{\supp}{supp}

% Zahlenmengen
\newcommand{\R}{\mathbb{R}}
\newcommand{\Q}{\mathbb{Q}}
\newcommand{\Z}{\mathbb{Z}}
\newcommand{\C}{\mathbb{C}}
\newcommand{\K}{\mathbb{K}}
\newcommand{\N}{\mathbb{N}}
\renewcommand{\P}{\mathbb{P}}
\newcommand{\F}{\mathbb{F}}

% Formatierung & Sonstiges 
\newcommand{\D}{\mathrm{d}}
\renewcommand{\phi}{\varphi}
\renewcommand{\epsilon}{\varepsilon}

\newcommand{\cali}[1]{\mathcal{#1}}
\usepackage{tikz-cd}
% header and footer with scrlayer-scrpage
\usepackage{scrlayer-scrpage}
\usepackage{fleqn}
\usepackage{xcolor}

\lohead*{Algebra 1}
\cohead*{}
\rohead*{Nils Witt}
\cofoot*{}
\lofoot*{Seite \thepage}
\pagestyle{scrheadings}
\title{Quizaufgabem}
\date{Wintersemester 2020}
\author{Nils Witt}
\setlength\parindent{0pt}

\begin{document}
\maketitle
\textbf{Aufgabe 1.} Sei $k$ ein Körper, $f=X^n-a\in k[X]$ mit $a\neq 0$, dann gilt
\begin{enumerate}[(i)]
    \item $f$ separabel genau dann, wenn $\operatorname{char}(k)\nmid n$
    \item Sind $\alpha,\beta$ Nullstellen von $f$, dann ist $\alpha/\beta$ eine $n$-te Einheitswurzel
\end{enumerate}
\begin{proof}
    (i): Es gilt $X^n-1\in k[X]$ separabel genau dann, wenn $\operatorname{char}(k)\nmid n$ und falls $X^n-1$ separabel, dann sind die Nullstellen von $f$ gerade $\zeta_n^i\sqrt[n]{a}\in \overline{k}$ für $i=0,\ldots,n-1$ mit einer primitiven $n$-ten Einheitswurzel $\zeta_n$.  
    Diese sind dann paarweise verschieden. Angenommen $\operatorname{char}(k)\mid n$, dann sind die Merhfachnullstellen von $f$ gerade die gemeinsamen Nullstellen von $f$ und $f'=nX^{n-1}=0$, das sind aber alle Nullstellen von $f$, also ist $f$ nicht separabel.
    (ii): Seien $\alpha,\beta$ Nullstellen von $X^n-a$, dann gilt 
    \[
    \left(\frac{\alpha}{\beta}\right)^n-1=\frac{\alpha^n}{\beta^n}-1=\frac{a}{a}-1=0    
    \]
\end{proof}
\textbf{Aufgabe 2.} Es gilt $[\Q(\sqrt[4]{6})/\Q(\sqrt{6})]=2$. 
\begin{proof}
    $\sqrt[4]{6}$ ist Nullstelle von $X^2-\sqrt{6}\in \Q(\sqrt{6}))[X]$, wäre $\sqrt[4]{6}\in \Q(\sqrt{6})$, dann gäbe es eine Darstellung mit $a,b\in \Q$, s.d.
    \[
        (a+b\sqrt{6})^2-\sqrt{6} = a^2+\sqrt{6}(2ab-1)+6b^2 = 0
    \] 
    gilt $2ab-1\neq 0$, dann können wir alles umschaffen und erhalten $\sqrt{6}\in \Q$. Widerpsruch. Wenn $2ab-1=0$, dann ist $a^2+6b^2=0$ und daher $a^2=-6b^2$, aber es ist $a^2,b^2\ge 0$, also Widerpsruch.
\end{proof}
\textbf{Aufgabe 3.} Behauptung: Die Körpererweiterungen $\Q(\sqrt[6]{2},\sqrt{18})/\Q,\Q(\sqrt[3]{2},e^{2\pi i/3})/\Q$ und $\Q(\sqrt{2+\sqrt[3]{2}})$ sind alle vom Grad 6.
\begin{proof}
    \begin{enumerate}[(i)]
        \item Es ist $[\Q(\sqrt[6]{2}):\Q]=6$ (betrachte: $X^6-2\in \Q[X]$ + Eisenstein) und wegen $\sqrt{18}=3\sqrt{2}$ und $(\sqrt[6]{2})^3=\sqrt{2}$ folgt die Behauptung.
        \item Es ist $[\Q(\sqrt[3]{2}):\Q]=3$ und $\Q(\sqrt[3]{2})\subsetneqq \R$ und $X^2+X+1\in \Q[X]$ ist das Mipo von $e^{2\pi i/3}$ über $\Q$ und alle Nullstellen davon sind komplex, daher $[\Q(e^{2\pi i/3},\sqrt[3]{2}):\Q(\sqrt[3]{2})]=2$. Gradsatz liefert Behauptung.
        \item $[\Q(\sqrt[3]{2}):\Q]=3$ und $(\sqrt{2+\sqrt[3]{2}})^2 - (2+\sqrt[3]{2})=0$.
    \end{enumerate}
\end{proof}
\end{document}