\textbf{Vorbemerkung:} Innerhalb dieses Vortrags sei $k=\Q_p$ mit $p$ einer Primzahl. Alle quadratischen Formen 
seien nicht ausgeartet. Zuerst erinnern wir uns an das Hilbertsymbol aus §3:
\begin{remind}[Hilbertsymbol]
    Sei $k\in\{\Q_p,\R\}$. Die Abbildung 
    \begin{align*}
        k^\times/(k^\times)^2\times k^\times/(k^\times)^2&\to \{0,1\} \\  (a,b) &\mapsto (a,b) \coloneqq 
        \begin{cases}
            1, & Z^2-aX^2-bY^2 \text{ hat Lösung } \neq (0,0,0)\text{ in } k^3  \\ 
            -1, & \text{sonst}
        \end{cases}
    \end{align*}
    ist eine nicht ausgeartete Bilinearform. Der Audruck $(a,b)$ heißt das \textrm{Hilbertsymbol} von $a$ und $b$ relativ zu $k$. Insbesondere haben wir 
    \begin{itemize}
        \item $(a,b)=(b,a)$ und $(a,c^2)=1$
        \item $(a,-a)=1$ und $(a,1-a)=1$
        \item $(a,b)=(a,-ab)=(a,(1-a)b)$
    \end{itemize}
    der Ausdruck \glqq nicht-ausgeartet\grqq{} soll in diesem Kontext heißen, dass $(a,b)=1$ für alle $b\in k$ impliziert $a$ ein Quadrat. 
\end{remind}
\subsection*{Die Invarianten quadratischer Formen}
Sei $(V,Q)$ ein quadratischer Modul und sei $\langle \cdot,\cdot\rangle$ die induzierte symmetrische Bilinearform, dann gibt es eine Orthogonalbasis von
$(V,Q)$, etwa $e=(e_1,\ldots,e_n)$ und $A=(\langle e_i,e_j\rangle)_{i,j}$, dann ist
\[
\operatorname{disc}(Q) \coloneqq \det A = a_1\cdot\ldots\cdot a_n     
\]
als Element von $k^\times/(k^\times)^2$ eindeutig bestimmt. 