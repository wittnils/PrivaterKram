\documentclass[a4paper,12pt]{scrartcl}
\usepackage{microtype}
\usepackage{geometry}
\usepackage[T1]{fontenc}
\usepackage[utf8]{inputenc}
\usepackage[ngerman]{babel}
% Standard-Packages

\usepackage[utf8]{inputenc}
\usepackage[T1]{fontenc}
\usepackage{amssymb}
\usepackage[fleqn]{amsmath}
\usepackage{amsfonts}
\usepackage{enumerate}
\usepackage{microtype}
\usepackage{extpfeil}
\usepackage{ngerman}
\usepackage{gauss}
\usepackage{mathtools}
\usepackage{mathrsfs}
\usepackage{xcolor}[dvipsnames]
\definecolor{MyBlue}{rgb}{0.2,0.2,0.7}


\usepackage{listofitems}
\newcommand\cycle[2][\,]{%
  \readlist\thecycle{#2}%
  (\foreachitem\i\in\thecycle{\ifnum\icnt=1\else#1\fi\i})%
}

% amsthm-Formatierung und Änderungen
\usepackage{amsthm}
\theoremstyle{plain}
\newtheorem{kor}{Korrollar}
\newtheorem{satz}{Satz}
\newtheorem{lemma}{Lemma}
\renewcommand*{\proofname}{Beweis}
\theoremstyle{remark}
\newtheorem*{bem}{\textbf{Bemerkung}}
\theoremstyle{definition}
\newtheorem{defn}{Definition}
\newtheorem*{defnstar}{Definition}

\newtheoremstyle{erinnerung}
{3pt}% Space above
{3pt}% Space below 
{}% Body font
{}% Indent amount
{\bfseries}% Theorem head font
{.}% Punctuation after theorem head
{.5em}% Space after theorem head
{}% Theorem head spec (can be left empty, meaning ‘normal’)

\theoremstyle{erinnerung}
\newtheorem*{remind}{Erinnerung}

% Lineare Algebra 
\newcommand{\spur}{\operatorname{Sp}}
\newcommand{\ann}{\operatorname{Ann}}
\newcommand{\charPol}[1]{\chi_{#1}^\text{char}}
\newcommand{\minPol}[1]{\chi_{#1}^\text{min}}
\newcommand{\Hom}[1]{\mathrm{Hom}_{#1}}
\newcommand{\End}[1]{\mathrm{End}_{#1}}
\newcommand{\Spv}[1]{\operatorname{Spv}(#1)}
\newcommand{\Spec}[1]{\operatorname{Spec}(#1)}
\newcommand{\RZ}[1]{\operatorname{RZ}(#1)}
\newcommand{\Xmeas}{(X,\mathscr{E},\mu)}
\newcommand{\Ymeas}{(X,\mathscr{F},\nu)}
\newcommand{\legendre}[2]{\left(\frac{#1}{#2}\right)}
\DeclareMathOperator{\kgV}{kgV}
\DeclareMathOperator{\ggT}{ggT}
\DeclareMathOperator{\GGT}{GGT}
\DeclareMathOperator{\im}{im}
\DeclareMathOperator{\rg}{Rang}
\DeclareMathOperator{\GL}{GL}
\DeclareMathOperator{\sgn}{sgn}
\DeclareMathOperator{\Lin}{Lin}
\DeclareMathOperator{\id}{id}
\DeclareMathOperator{\esssup}{\operatorname{ess \ sup}}
\DeclareMathOperator{\pr}{pr}
\DeclareMathOperator{\gal}{Gal}
\DeclareMathOperator{\ord}{ord}
\DeclareMathOperator{\supp}{supp}

% Zahlenmengen
\newcommand{\R}{\mathbb{R}}
\newcommand{\Q}{\mathbb{Q}}
\newcommand{\Z}{\mathbb{Z}}
\newcommand{\C}{\mathbb{C}}
\newcommand{\K}{\mathbb{K}}
\newcommand{\N}{\mathbb{N}}
\renewcommand{\P}{\mathbb{P}}
\newcommand{\F}{\mathbb{F}}

% Formatierung & Sonstiges 
\newcommand{\D}{\mathrm{d}}
\renewcommand{\phi}{\varphi}
\renewcommand{\epsilon}{\varepsilon}

\newcommand{\cali}[1]{\mathcal{#1}}
\usepackage{tikz-cd}
% header and footer with scrlayer-scrpage
\usepackage{scrlayer-scrpage}
\usepackage{fleqn}
\usepackage{xcolor}

\lohead*{Höhere Analysis}
\rohead*{Nils Witt}
\cofoot*{}
\lofoot*{Seite \thepage}
\pagestyle{scrheadings}
\title{Integrationstheorie in der Analysis 3}
\date{Wintersemester 2020}
\author{}
\setlength\parindent{0pt}

\begin{document}
\maketitle
Seien stets $(X,\mathscr{A},\mu)$ und $(Y,\mathscr{F},\nu)$ ein Maßräume.
\section{Konvergenzsätze: Lebesgue-Integral, $L^p$-Räume}
Wir definieren noch einmal die grundlegenden Begriffe.
    \begin{defn}[$\mathscr{L}^p$-Räume] Sei $f:X\to \R$ messbar, wir definieren 
        \[
        \Vert f\Vert_p \coloneqq \left(\int_X \vert f\vert^p \D \mu\right)^{1/p} \quad \mathrm{mit } \ 1\le p<\infty    
        \]
        und wir setzen $\Vert f\Vert_\infty \coloneqq \esssup_X \vert f\vert$. Dann definieren wir für $1\le p\le \infty$
        \[
        \mathscr{L}^p(X,\mu) = \{f:X\to \R \ \mathrm{messbar} \mid \Vert f\Vert_p<\infty\}    
        \]
    \end{defn}
    Dass $\Vert \cdot\Vert_p$ auf $\mathscr{L}^p(X)$ keine Norm ist, liegt daran, dass 
    nicht $\Vert f\Vert_p = 0 \iff f=0$ gilt. Wir lösen dieses Problem.\\
    Sei $\sim$ eine Äquivalenzrelation auf $\mathscr{L}^p(X)$ gegeben durch $f\sim g\iff f=g \ \mu\text{-fast-überall}$. Dann definieren wir
    \[
    L^p(X,\mu) \coloneqq \mathscr{L}^p(X,\mu)/\sim    
    \] 
     Zuerst zitieren wir die wichtigsten 
    Konvergenzsätze für das Lebesgue-Integral
    \begin{satz}[Satz von der monotonen Konvergenz] Sei $f_k:X\to \R, \ k\in \N$ eine Folge von Funktionen mit 
    \begin{enumerate}
        \item $f_k$ messbar für alle $k\in \N$
        \item $f_k$ monoton wachsend und $f_k \to f$ für $k\to \infty$
    \end{enumerate} Dann gilt, dass $f$ messbar ist und
        \[
        \lim_{k\to \infty} \int_X f\ \D \mu = \int_X \lim_{k\to \infty} f\ \D \mu    
        \]
    \end{satz}
    \begin{satz}[Satz von der dominierten Konvergenz] Sei $f_k:X\to \R$ messbar. Ferner gelte ${\sup_k \Vert f_k\Vert \le g}$ für eine integrierbare Funktion $g:X\to \R$, dann gilt
        \[
        \lim_{k\to \infty} \int_X f_k \ \D \mu = \int \lim_{k\to\infty } f_k \ \D \mu    
        \]
    \end{satz}
    \begin{bem}[Monotone Konvergenz in $L^p$] Sei $1\le p<\infty$ und sei 
        $f_n\in L^p(X)$ für $n\in \N$, dann gelte
        \begin{enumerate}
            \item $f_n \to f$ $\mu$-fast-überall. 
            \item $f_n \le g, \ \forall n\in \N$ und $\mu$-fast-überall für ein $g\in L^p(X)$
        \end{enumerate}
        Dann gilt $f\in L^p(X)$ und $f_k \to f$ in $L^p(X)$.
    \end{bem}
    \textcolor{red}{\textbf{Gegenbeispiel für $p=\infty$:} Betrachte $(\R, \mathscr{L}^1, \lambda)$. Sei $f_k=\chi_{[k,k+1]}$, dann ist $f_k\to 0$ punktweise, es ist $f_k\le 1=g$ und $g\in L^\infty(X)$
    , aber es ist auch 
    \[
    \Vert f_k - 0\Vert_\infty = \Vert f_k\Vert_\infty = 1, \ \forall k\in \N    
    \]
    also gilt nicht $f_k\to 0$ in $L^\infty(X)$.}
    \begin{lemma}[Hölder-Ungleichung] Sei $f\in \mathscr{L}^p(X)$ und $g\in\mathscr{L}^{p_*}(X)$ mit $p^{-1}+p_*^{-1}=1$ 
        dann gilt 
        \[
        \Vert fg\Vert_1 = \int_X \vert fg\vert \ \D \mu \le \Vert f\vert_p \cdot \Vert g\Vert_{p_*}     
        \]
    \end{lemma}
    \begin{lemma}[Minkowski-Ungleichung] Für $1\le p\le \infty$ seien $f,g\in \mathscr{L}^p(X)$, dann gilt 
        \[
        \Vert f+g\Vert_p \le \Vert f\Vert_p + \Vert g\Vert_p    
        \]
        Dadurch wird $\Vert\cdot\Vert_p$ auf $\mathscr{L}^p(X)$ zu einer Seminorm.
    \end{lemma}
    \section{Vergleich der Konvergenzbegriffe}
    \begin{defn}[We're getting closer] Wir führen die Konvergenzbegriffe alle auf. Sei dazu $f_n:X\to \R$ eine Folge von Funktionen und $f:X\to \R$.
        \begin{enumerate}
            \item Wir sagen $f_n \to f$ konvergiert gleichmäßig, falls
            \[
            \sup_{x\in X} \vert f_n(x)-f(x)\vert \xrightarrow{n\to \infty} 0    
            \] 
        \item Wir sagen $f_n$ konvergiert punktweise gegen $f$, falls 
        \[
        f_n(x) \xrightarrow{n\to \infty} f(x), \ \forall x\in X    
        \]
        \item $f_n$ konvergiert gegen $f$ punktweise $\mu$-fast-überall, falls 
        \[
        \exists N\subset X:  \mu(N)= 0 \ \text{und} \ f_n(x)\xrightarrow{n\to \infty} f(x), \ \forall x\in X\setminus N    
        \]
        \item $f_n$ konvergiert im Maß $\mu$, falls $f_n,f$ messbar und für $\epsilon>0$ beliebig gilt
        \[
        \mu(\{x\in X : \vert f_k(x)-f(x)\vert > \epsilon\}) \xrightarrow{k\to \infty} 0    
        \]
        \item Für $f_n\in L^p(X)$ sagen wir, dass $f_n$ in $L^p(X)$ konvergiert, falls
        \[
        \Vert f_n - f\Vert_p \xrightarrow{n\to \infty} 0    
        \]
        Insbesondere folgt dann $f\in L^p(X)$, da $\mathscr{L}^p(X)$ ein Banachraum ist.
        \end{enumerate}
    \end{defn}
    Jetzt vergleichen wir die Konvergenzbegriffe. Seien $f_n,f:X\to \R, \ n\in \N$ messbar. 
    \begin{enumerate}
        \item Falls $f_k \to f$ im Maß, dann existiert eine Teilfolge $f_{kj}$ mit $f_{kj}\to f$ $\mu$-fast-überall.
        \textcolor{red}{Gilt die Umkehrung? Im Allgemeinen können wir auch nicht sagen, dass $f_n\to f$ punktweise, denn:\\
        Sei $f_1=\chi_{[0,1]}$ und wir definieren $f_{kj} = \chi_{[j/k,(j+1)/k]}, \ j=0,\ldots,k-1$. Dann nummerieren wir die $f_{kj}$ als $f_n$ und es gilt $f_n\to 0$ im Lebensguemaß, aber es gilt nicht $f_n \to 0$ punktweise.}
        \item Falls \textbf{$\mu(X)<\infty$}, dann gilt: $f_k\to f$ $\mu$-fast-überall, so folgt $f_k\to f$ im Maß.
        \textcolor{red}{\\ Im Allgemeinen gilt das nicht. Sei $X=\R$ mit Lebensgue-Maß. Sei $f_k=\chi_{[k,k+1]}$, dann ist $f_k\to 0$ punktweise, also insbesondere punktweise $\mu$-fast-überall, aber es ist
        \[
        \mu(\{x\in X : \vert f_k(x)\vert > \epsilon\}) = \begin{cases} \mu([k,k+1]=1, & \epsilon<1\\
        0, & \text{sonst} \end{cases}    
        \] Also liegt für $\epsilon <1$ keine Konvergenz im Maße vor.}
    \end{enumerate}
    \textbf{Endliche Maßräume:}\begin{itemize}
        \item[--] Gleichmäßige Konvergenz $\implies$ Konvergenz in $L^p$, für alle $p\in[1,\infty] \implies$ Konvergenz im Maß
        \item[--] Konvergenz in $L^p$ $\implies$ Konvergenz in $L^q$ für $1\le q\le p$   
    \end{itemize}
    Insbesondere letzteres gilt \textbf{nicht} in allgemeinen Maßräumen!\\ 
    \textbf{Allgemeine Maßräume:}
    \begin{itemize}
        \item[--] Konvergenz im Maß ist schwächer als jede $L^p$-Konvergenz, denn es gilt
        \[
            \Vert f_n-f\Vert_p \to 0 \ \text{für ein} \ p\in[1,\infty] \implies f_n \to f \ \text{im Maß} \ \mu 
        \]
        \item[--] Gleichmäßige Konvergenz $\implies$ punktweise Konvergenz $\implies$ punktweise fast-überall Konvergenz
        \item[--] Gleichmäßige Konvergenz $\implies$ Konvergenz in $L^\infty$ $\implies$ punktweise fast-überall
    \end{itemize}
    \newpage 
    \section{Fubini, Tonelli und Transformationsformel}
    \begin{lemma}[Tonelli] Seien $\Xmeas, \Ymeas$ hier $\sigma$-endlich. Sei \(f:X\times Y \to [0,\infty]\) bezüglich $\mathscr{E}\times\mathscr{F}$ messbare und \textbf{nicht-negative}
        Funktionen, dann gilt
        \begin{align*}
            \int_{X\times Y} f(x,y)  \D (\mu \times \nu) &= \int_X\left( \int_Y f(x,y)  \D \nu(y)\right) \D \mu(x) \\ &= \int_Y\left( \int_X f(x,y)  \D \mu(x)\right) \D \mu(y)
        \end{align*}
        Falls also eine Funktion bezüglich der Produk-$\sigma$-Algebra zweier Maßräume messbar ist, können wir die Integrationsreihenfolge vertauschen.
    \end{lemma}
    \textcolor{blue}{\textbf{Beispiel:} Sei $f:[0,1]^2\to \R, \ f(x,y)=2x^4y^2$, dann ist $f$ nichtnegativ und stetig, also insbesondere Borel-messbar, also gilt 
    \begin{align*}
        \int_{[0,1]^2} 2x^4 y^2 \ \D \mathscr{L}^2 &= \int_{[0,1]}\int_{[0,1]} 2x^4 y^2 \ \D y \D x \\
        & = \int_{[0,1]} 2x^4 \int_{[0,1]} y^2 \D y\D x = \int_{[0,1]} 2x^4 \left[\frac{1}{3}y^3\right]_0^1 \D x\\
        &= \int[0,1] 2x^4 \cdot \frac{1}{3} \D x = \frac{2}{3}\int_{[0,1]} x^4 \D x = \frac{2}{3}\cdot \frac{1}{5} = \frac{2}{15}
    \end{align*} }
    \begin{satz}[Fubini] 
        Seien $\Xmeas, \Ymeas$ hier $\sigma$-endlich. Sei \(f:X\times Y \to \R\) bezüglich $\mathscr{E}\times\mathscr{F}$ messbar, dann gilt
        \begin{align*}
            \int_{X\times Y} \vert f(x,y)\vert \D (\mu \times \nu) &= \int_X \left(\int_Y \vert f(x,y)\vert \D \nu(y)\right)\D \mu(x) \\ 
             &= \int_Y \left(\int_X \vert f(x,y) \vert \D \mu(x) \right)\D \nu(y) 
        \end{align*}
        Ist insbesondere eines der Integrale endlich, dann können die Betragsstriche weggelassen werden.
    \end{satz}
    \begin{satz}[Transformationsformel] Seien $U,V\subset \R^n$ offen und $F\in L^1(V)$ und sei $\phi\in C^1(U,V)$ ein $C^1$-Diffeomorphismus, dann gilt
        \[
        \left(\int_V F \ \D x =\right)\int_{\phi(U)} F \ \D x = \int_U (F\circ \phi) \vert \det (D\phi)\vert \ \D x    
        \]
        
    \end{satz}
\end{document}