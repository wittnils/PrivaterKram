Sei $(\Gamma,\le)$ eine total geordnete (multiplikativ geschriebene) abelsche Gruppe, $A$ ein Ring. 
\subsection*{Wiederholung}
Eine \textit{Bewertung} $\vert \cdot \vert:A\to \Gamma \cup \{0\}$ ist eine Abbildung mit 
\begin{enumerate}[(i)]
   \item $\vert a+b\vert \le \max\{ \vert a\vert, \vert b\vert \}$ für alle $a,b\in A$
    \item $\vert ab\vert = \vert a\vert \vert b \vert$ für alle $a,b\in A$
    \item $\vert 0 \vert =0$ und $\vert 1\vert =1$
\end{enumerate}
die Menge $\vert \cdot \vert^{-1}(\{0\})\eqqcolon \operatorname{supp}(\vert \cdot \vert)$ heißt der \textit{Träger} von $\vert \cdot \vert$ und die Untergruppe in $\Gamma$, die von $\im (\vert \cdot \vert) \setminus \{0\}$ erzeugt wird, heißt Wertegruppe von $\vert \cdot \vert$ und wird 
mit $\Gamma_{\vert\cdot \vert}$ bezeichnet. \\
Zwei Bewertungen $\vert \cdot \vert_1, \ \vert \cdot \vert_2$ heißen äquivalent, wenn eine der äquivalenten Bedingungen erfüllt ist 
\begin{enumerate}[(i)]
    \item Es gibt einen Isomorphismus total georndeter Monoide $f:\Gamma_{\vert \cdot \vert_1}\cup \{0\}\to \Gamma_{\vert \cdot\vert_2}\cup \{0\}$ mit $f\circ \vert \cdot\vert_1 = \vert \cdot \vert_2$
    \item $\supp (\vert\cdot\vert_1 )= \supp( \vert \cdot\vert_2)$ und $A(\vert\cdot\vert_1) = A(\vert \cdot\vert_2)$
    \item Es gilt $\vert a\vert_1 \le \vert b\vert_1 \Leftrightarrow \vert a\vert_2 \le \vert b\vert_2$ für alle $a,b\in A$.
\end{enumerate}
\subsection*{Das Bewertungsspektrum eines Ringes}
\begin{defn}
    Das \textit{Bewertungsspektrum} von $A$ ist die Menge aller Äquivalenzklassen von Bewertungen auf $A$ und wird 
    mit $\Spv{A}$ bezeichnet. Man betrachte für $f_1,\ldots,f_n,g\in A$
    \[
    U(\frac{f_1,\ldots,f_n}{g}) = \{ \vert \cdot \vert \in \Spv{A} : \vert f_i\vert \le \vert g\vert \neq 0, \ \forall i=1,\ldots,n\}    
    \]
    dann gilt für $f,f',g,g'\in A$, dass 
    \[
    U(\frac{f}{g})\cap U(\frac{f'}{g'}) = U(\frac{fg',f'g}{gg'})    
    \]
    \textit{Grund}: Es ist $\vert gg'\vert = \vert g\vert \vert g'\vert$ und daher $\vert gg'\vert \neq 0$ äquivalent zu $\vert g\vert,\vert g'\vert \neq 0$. $\vert fg' \vert \le \vert gg'\vert$ und $\vert f'g \vert \le \vert gg'\vert$ sind wegen $\vert g\vert,\vert g'\vert \neq 0$ äquivalent zu $\vert f\vert \le \vert g\vert$ und $\vert f'\vert \le \vert g'\vert$.
    \\ Daher erzeugen die Mengen $U((f_1,\ldots,f_n)/g)$ mit $f_1,\ldots,f_n,g\in A$ eine Topologie auf $\Spv{A}$.
\end{defn}
Man hat eine Bijektion 
\[
    X\coloneqq\{  (\mathfrak{p},R) \mid \mathfrak{p}\in \operatorname{Spec}(A), \ R \text{ Bur von } \operatorname{Frac}(A/\mathfrak{p}) \} \longleftrightarrow \Spv{A}    
\]
Inutuitiv: $R$ kodiert den Bereich, in dem die Bewertung $\le 1$ ist und $\mathfrak{p}$ den, wo sie $=0$ ist, so kann die gesamte Bewertung rekonstruiert werden. Sei $(\mathfrak{p},R)\in X$ auf $\kappa(\mathfrak{p})=\operatorname{Frac}(A/\mathfrak{p})$ haben wir die Bewertung
\[
\kappa(\mathfrak{p}) \to \kappa(\mathfrak{p})^\times / R^\times \cup \{0\}, \ x\mapsto \begin{cases}
    xR^\times & x\neq 0 \\
    0 & x=0
\end{cases}    
\]
und kanonische Homomorphismen $\phi^\mathfrak{p}:A\to A/\mathfrak{p} \to \kappa(\mathfrak{p})$ dadurch erhalten wir dann eine Bewertung auf $A$. Diese sieht dann konkret aus 
\[
\vert \cdot \vert_\mathfrak{p}^R : A\to \kappa(\mathfrak{p})^\times / R^\times \cup\{0\}, x \mapsto \begin{cases}
    \frac{x+\mathfrak{p}}{1+\mathfrak{p}}R^\times, & x\notin \mathfrak{p} \\
    0, & x\in \mathfrak{p}
\end{cases}  
\]
Ist andererseits $\vert \cdot \vert\in \Spv{A}$, dann ist $\mathfrak{p}\coloneqq \supp(\vert \cdot \vert)\in \operatorname{Spec}(A)$ (denn: $x,y\notin \mathfrak{p}$ und sei $\Gamma$ die Zielgruppe von $\vert\cdot\vert$ dann $\vert x\vert,\vert y\vert\in \Gamma$, daher $\vert x\vert \vert y\vert = \vert xy\vert \in \Gamma$, insb. $\neq 0$) und $\vert \cdot\vert$ definiert eine Bewertung auf $A/\mathfrak{p}$, denn seien $x,x'\in A$ mit $x-x'\in \mathfrak{p}$, dann gilt $\vert x\vert \le \max\{ \vert x-x'\vert,\vert x'\vert \}=\vert x'\vert$ und umgekehrt. Diese Bewertung auf $A/\mathfrak{p}$ setzt sich dann auf $\operatorname{Frac}(A/\mathfrak{p})$ fort,
nämlich durch 
\[
\vert\cdot\vert':\kappa(\mathfrak{p}) \to \Gamma \cup \{0\}, \ \frac{x+\mathfrak{p}}{y+\mathfrak{p}}\mapsto \left\vert \frac{x+\mathfrak{p}}{y+\mathfrak{p}}\right\vert'=\vert x\vert \cdot \vert y\vert^{-1}     
\]
dies ist wohldefiniert, seien $x,x',y,y'\in A$ mit 
\[
\frac{x+\mathfrak{p}}{y+\mathfrak{p}} = \frac{x'+\mathfrak{p}}{y'+\mathfrak{p}}    
\]
also mit $xy'-x'y\in \mathfrak{p}$. Ist $\vert x'\vert =0$, also $x'\in \mathfrak{p}$, so gilt $xy'\in \mathfrak{p}$, da aber $y'\notin \mathfrak{p}\in \Spec{A}$, folgt $x\in \mathfrak{p}$, also $\vert x\vert =0$. Ist $\vert x'\vert \neq 0$, so folgt die Aussage 
\[
1 = \vert x'y\vert \vert x'y\vert^{-1} = \vert xy'\vert \vert x'y\vert^{-1} \iff \vert x'\vert\vert y'\vert^{-1} = \vert x\vert \vert y\vert^{-1}     
\]
Man kann zeigen, dass diese Bildungen invers zueinander sind. 
\begin{remind}[Riemann-Zariski-Räume] Sei $A$ ein Ring, $k$ ein Körper mit $A\subset k$, dann heißt 
    \[
    \RZ{k,A} = \{ R \mid R\text{ ist Bewertungsunterring}:A\subset R\subset k\}    
    \]
    der \textit{Riemann-Zariski-Raum} von $k$ über $A$, ist $A$ das Bild von $\Z$ in $k$, so heißt 
    \[
    \RZ{k,A} = \RZ{k} = \{ R \mid R\text{ ist Bewertungsunterring}:R\subset k\}     
    \]
    \textit{der} Riemann-Zariski-Raum von $k$. Für $x_1,\ldots,x_n\in k$, sei 
    \[
    U(x_1,\ldots,x_n) \coloneqq \RZ{k,A[x_1,\ldots,x_n]} = \{ R\in \RZ{k,A} : x_1,\ldots,x_n\in R\}    
    \]
    und wegen $U(x_1,\ldots,x_n)\cap U(y_1,\ldots,y_m) = U(x_1,\ldots,x_n,y_1,\ldots,y_m)$ können wir diese Mengen zur Definition
    einer Topologie verwenden.
\end{remind}
\begin{lemma}[Beschreibung der Fasern als RZ-Räume]
    Sei $A$ ein Ring, dann gilt 
    \begin{enumerate}[(i)]
        \item Ist $A$ ein Körper, so gilt $\Spv{A} \cong_{\mathsf{Top}} \operatorname{RZ}(A)$
        \item Die kanonische Abbildung $\supp : \Spv{A} \to \Spec{A},\ \vert\cdot\vert \mapsto \supp(\vert\cdot\vert)$ ist stetig (bzgl. der Zariski-Topologie auf $\Spec{A}$) und surjektiv. Für jedes $\mathfrak{p}\in \Spec{A}$ ist dessen Faser unter $\supp$ als topologischer Raum zu $\RZ{\operatorname{Frac}(A/\mathfrak{p})}$ isomorph.
    \end{enumerate}
\end{lemma}
\begin{proof}
    (ii): Sei $\mathfrak{p}\in \Spec{A}$, dann betrachte die triviale Bewertung $\vert\cdot\vert_{\mathfrak{p},\text{triv}}\in \Spv{A}$, dann gilt $\supp(\vert\cdot\vert_{\mathfrak{p},\text{triv}})=\mathfrak{p}$, daher ist $\supp$ surjektiv.  
    Die Mengen $D(f)=\{\mathfrak{p}\in \Spec{A} : f\notin \mathfrak{p}\}$ bilden eine Basis der Zariski-Topologie, es gilt 
    \[
    \supp^{-1}(D(f)) = \{ \vert\cdot\vert\in \Spv{A} : f\notin \supp(\vert\cdot\vert) \} =U(\frac{f}{f})   
    \]
    daher ist $\supp$ stetig. Mit Wir haben eine Bijektion
    \begin{align*}
        \Spv{A} &\longleftrightarrow\{ (\mathfrak{p},R) \mid \mathfrak{p}\in \Spec{A}, R\in \RZ{\kappa(\mathfrak{p})}\} \\
        \vert\cdot\vert &\mapsto (\supp(\vert\cdot\vert),R_{\vert\cdot\vert})
    \end{align*}
    wobei $R_{\vert\cdot\vert}$ gegeben ist als 
    \[
        R_{\vert\cdot\vert} = \left\{\frac{x+\mathfrak{p}}{y+\mathfrak{p}}\in \kappa(\mathfrak{p}) : \vert x\vert \vert y\vert^{-1}\le 1\right\}
    \]
    diese induziert eine Bijektion
    \[
    \Psi: \supp^{-1}(\{\mathfrak{p}\}) \cong_{\mathsf{Set}} \{ (\mathfrak{p},R) : R\in \RZ{\kappa(\mathfrak{p})}\} \cong_{\mathsf{Set}} \RZ{\kappa(\mathfrak{p})}, \quad \vert\cdot\vert\in \Spv{A} \mapsto R_{\vert\cdot\vert}
    \]
    wir wollen zeigen, dass $\supp^{-1}(\{\mathfrak{p}\}) \cong_{\mathsf{Top}}\RZ{\kappa(\mathfrak{p})}$, wobei $\supp^{-1}(\{\mathfrak{p}\})$ mit der Teilraumtpologie von $\Spv{A}$ ausgestattet wird. Wir zeigen, dass sie stetig und offen ist.
    \begin{enumerate}[(i)]
        \item Seien $x_1,\ldots,x_n\in A, y\in  A \setminus \mathfrak{p}$. Dann gilt 
        \[
        \vert\cdot\vert\in \Psi^{-1}(U(\frac{x_1+\mathfrak{p}}{y+\mathfrak{p}},\ldots,\frac{x_n+\mathfrak{p}}{y+\mathfrak{p}})) \Leftrightarrow \vert x_i \vert \vert y\vert^{-1} \le 1,\ \forall i=1,\ldots,n \Leftrightarrow \vert\cdot\vert \in U(\frac{x_1,\ldots,x_n}{y})
        \]
        also ist $\Psi$ stetig. 
        \item Seien $f_1,\ldots,f_n,g \in A$ und betrachte $U\coloneqq U(\frac{f_1,\ldots,f_n}{g})$ beliebige offene Menge in $\Spv{A}$, ist $g\in \mathfrak{p}$, so ist $U\cap \supp^{-1}(\mathfrak{p})=\varnothing$. \\
        Sei also $g\notin \mathfrak{p}$, dann ist
        \[
        U \cap \supp^{-1}(\{ \mathfrak{p} \}) = \{ \vert\cdot\vert \in \Spv{A} : \vert f_i \vert \vert g\vert^{-1} \le 1,\ i=1,\ldots,n\}    
        \]
        daher gilt 
        \begin{align*}
            \Psi(U\cap \supp^{-1}(\{\mathfrak{p}\})) &= \{ R \in \RZ{\kappa(\mathfrak{p})} : (f_i+\mathfrak{p})(g+\mathfrak{p})^{-1} \in R,\ \forall i=1,\ldots,n\} 
            \\ &= U (\frac{f_1+\mathfrak{p}}{g+\mathfrak{p}},\ldots,\frac{f_n+\mathfrak{p}}{g+\mathfrak{p}})     
        \end{align*}
        die Inklusion $\subset$ in der ersten Gleichheit ist klar. Sei $R\in \RZ{\kappa(\mathfrak{p})}$ mit $(f_i+\mathfrak{p})(g+\mathfrak{p})^{-1}\in R$ für $i=1,\ldots,n$. 
        So betrachte die Bewertung 
        \[
            v:\kappa(\mathfrak{p}) \to \kappa(\mathfrak{p})^\times/R^\times \cup\{0\}, x \mapsto \begin{cases}
                xR^\times, & x\neq 0 \\
                0, & x=0
            \end{cases}
        \]
        und sei $w=v\circ (A\to \kappa(\mathfrak{p}))$, dann gilt $w(f_i)\le w(g)\neq 0, \ i=1,\ldots,n$, denn: $w(g)\neq 0$, denn $\ker(A\to \kappa(\mathfrak{p}))=\mathfrak{p}$ und $g\notin \mathfrak{p}$ und $v$ ist eine Bewertung auf einem Körper. Ist $f_i \in \mathfrak{p}$, so ist die Behauptung klar. Also $f_i \notin \mathfrak{p}$, dann 
        \[
        w(f_i) = \frac{f_i+\mathfrak{p}}{1+\mathfrak{p}}R^\times \le \frac{g+\mathfrak{p}}{1+\mathfrak{p}}R^\times =w(g) \Leftrightarrow \frac{f_i+\mathfrak{p}}{1+\mathfrak{p}}\frac{1+\mathfrak{p}}{g+\mathfrak{p}} = \frac{f_i+\mathfrak{p}}{g+\mathfrak{p}} \in R    
        \]
        also $w\in U\cap \supp^{-1}(\{\mathfrak{p}\})$, wie im Appendix sieht man, dass $R_w  = R$ gilt, daher folgt die Aussage.
    \end{enumerate}
    da stetige, offene Abbildungen Isomorphismen in $\mathsf{Top}$ sind, folgt die Aussage. \\
    (i): Ist $A=k$ ein Körper, so ist $\Spec{k} = \{ (0) \}$, daher auch $\operatorname{Frac}(k/(0)) =k$, daher $\supp^{-1}(\{ 0\}) = \Spv{k} \cong_{\mathsf{Top}} \RZ{k}$. 
\end{proof}