\newpage
\subsection*{Funktorialität von $\mathsf{Spv}$}
Seien $A,B$ zwei Ringe (kommutativ, mit $1$) und $\phi:A\to B$ ein Ringhomomorphismus, d.h. $B$ ist eine $A$-Algebra. 
Dann haben wir die induzierte Abbildung 
\[
\Spv{\phi}:\Spv{B} \to \Spv{A}, \quad v \mapsto v\circ \phi    
\]
es für $f_1,\ldots,f_n,g\in A$:
\[
\Spv{\phi}^{-1}\left(U\left(\frac{f_1,\ldots,f_n}{g}\right)\right) = U\left(\frac{\phi(f_1,\ldots,f_n)}{\phi(g)}\right)    
\]
daher ist $\Spv{\phi}$ stetig. Es ist klar, dass $\Spv{\id_A} = \id_{\Spv{A}}$ und $\Spv{\psi\circ \phi} = \Spv{\phi} \circ \Spv{\psi}$ für gewisse Ringhomomorphismen $\phi,\psi$. 
Daher haben wir einen kontravarianten Funktor 
\begin{align*}
    \operatorname{Spv} : \mathsf{CRing}^{\mathsf{op}} &\to \mathsf{Top}    \\
    A &\mapsto \Spv{A} \\
    [\phi:A\to B] &\mapsto \Spv{\phi}    
\end{align*}
